\documentclass[a4paper,11pt]{article}
	\usepackage[]{mcode}
    \usepackage[utf8]{inputenc}
    \usepackage[T1]{fontenc} 
    \usepackage[french]{babel}
    \usepackage{graphicx}
    \usepackage{anysize}
    \usepackage{amsfonts}
    \usepackage{subcaption}
    \marginsize{2cm}{2cm}{2cm}{2cm}
    \title{Simulation d'une chaîne de transmission numérique sur canal gaussien à bande limitée\newline [Rapport]}
    \author{Jorge GULFO MONSALVE\\Aquiles TORRES ALVAREZ}

\begin{document}
\maketitle

\section{Géneration aléatoire des éléments binaires}
Dans cette partie on veut générer un vecteur aléatoire $b_n$ de taille $N=2048$. Pour des raisons de modularité on a créé une fonction \emph{bit\_generator} qui prend comme paramètre $N$ et donne le vecteur de bits aléatoires $b_n$ plus ses moyenne ($m_{emp}$) et variance ($sigma2_{emp}$) empirique. 

\section{Conversion des éléments binaires en symboles \emph{(mapping)}}
Maintenant, il faut faire la conversion des éléments binaires $b_n$ en symboles. En sachant que la modulation est 2-PAM on peut creér le vecteur $a_k$ en utilisant la relation.
\[ a_k = 2b_n - 1\]
La fonction \emph{mapping\_2PAM } fait cette conversion. La costellation des symboles est dans la figure \ref{fig:sec1}.
\begin{figure}
	\begin{center}
	\includegraphics[scale=1]{contesllation_a_k-crop.pdf}
	\caption{Contellation des symboles $a_k$}
	\label{fig:sec1}
	\end{center}
\end{figure} 

\section{Conversion numérique - analogique}

\subsection{Expansion}
Jusqu'à maintenant les symboles sont des valeurs ponctuels dans le vecteur $a_k$. Pour "Mettre en forme" des symboles il faut les associer un signal analogique, ça n'est pas possible dans une simulateur numérique , mais on peut simuler l'aspect d'un signal continu en utilisant des sur-échantillonnage par symbole. Dans notre cas, le facteur de sur-échantillonnage $F$ est $16$, c'est-à-dire que chaque symbole vas être représenté par la valeur du symbole plus $F-1=15$ zéros.

\subsubsection*{Question 1}
L'operation d'expansion est fait à la fonction \emph{expansion} et la figure \ref{fig:ques1} montre le résultat obtenu  en temps.

\begin{figure}
	\begin{center}
	\includegraphics[scale=1]{expansion-crop.pdf}
	\caption{Expansion des symboles $a_k$}
	\label{fig:ques1}
	\end{center}
\end{figure} 

\subsection{Étude des filtres}

\subsubsection*{Question 2}
Jusqu'ici on a les symboles expansés $a_k$. On ne peut pas transmettre cet signal, premièrement il faut le donner une "forme". Cela est le travail des filtres d'émission, ils donnent à la chaine de bits à transmettre ses formes selon le type de modulation et les paramètres de transmission. Nous sommes intéressés par les filtres rectangulaires (NRZ et RZ) et les filtres de cosinus surélevé (SRRC), donc on fera premièrment une analyse temporelle et en fréquence de ces filtres de mise en forme.\\ \\

Por le système de base ($D=1MHz, \alpha=0.5$) la réponse impulsionnelle et fréquentielle des filtres NRZ, RZ et SRRC se trouvent dans les figures \ref{fig:nrz1M}, \ref{fig:rz1M} et \ref{fig:srrc1M} respectivement. Tous les réponses impulsionnelles ont une durée de $8 t_b$ comme on a fixé ($L=8$), en plus dans le case des filtres rectangulaires $T_b$ est évident ($T= 1/D$ pour NRZ et $T/2$ pour RZ) et pour le filtre de cosinus surélevé on peut regarde que le signal est 0 chaque $T$. \\ \\

D'autre part, ce qui concerne à la réponse fréquentielle des filtres on peut les regarder à la partie (b) de chaque figure. Pour les filtres rectangulaires on estime sa bande passante grâce au lobe principal (ce qui porte la plupart d'information), donc le filtre RZ a une bande passante deux foies ($R=2D$) plus grand que le filtre NRZ ($R=D$). Le filtre SRRC a la bande passante plus petite que les autres à cause du coefficient d'excès de bande $\alpha$ qui est $0.5$, on parlera sur cette relation plus tard.\\ \\

Quand on change le débit binaire la réponse impulsionnelle et fréquentielle change. Si le débit est plus petit $T_b$ augmente et la bande passante diminue. Si c'est le case contraire $T_b$ diminue et la bande passante augmente. Dans les figures \ref{fig:nrz15M}, \ref{fig:rz15M} et \ref{fig:srrc15M} on voie une exemple de cas où $T_b$ augmente à 15Mbit/s.\\ 

\begin{figure}
	\begin{subfigure}{.5\textwidth}
  		\centering
  		\includegraphics[width=1\linewidth]{impul_nrz-crop.pdf}
  		\caption{Réponse impulsionnelle}
  		\label{fig:nrz_impul1M}
	\end{subfigure}
	\begin{subfigure}{.5\textwidth}
  		\centering
  		\includegraphics[width=1\linewidth]{frec_nrz-crop.pdf}
  		\caption{Réponse fréquentielle}
  		\label{fig:nrz_frec1M}
	\end{subfigure}%
	\caption{Réponse impulsionnelles et fréquentielles du filtre NRZ - $D=1 Mbit/s$}
	\label{fig:nrz1M}
\end{figure} 

\begin{figure}
	\begin{subfigure}{.5\textwidth}
  		\centering
  		\includegraphics[width=1\linewidth]{impul_rz-crop.pdf}
  		\caption{Réponse impulsionnelle}
  		\label{fig:rz_impul1M}
	\end{subfigure}
	\begin{subfigure}{.5\textwidth}
  		\centering
  		\includegraphics[width=1\linewidth]{frec_rz-crop.pdf}
  		\caption{Réponse fréquentielle}
  		\label{fig:rz_frec1M}
	\end{subfigure}%
	\caption{Réponse impulsionnelles et fréquentielles du filtre RZ - $D=1 Mbit/s$}
	\label{fig:rz1M}
\end{figure}

\begin{figure}
	\begin{subfigure}{.5\textwidth}
  		\centering
  		\includegraphics[width=1\linewidth]{impul_srrc-crop.pdf}
  		\caption{Réponse impulsionnelle}
  		\label{fig:srrc_impul1M}
	\end{subfigure}
	\begin{subfigure}{.5\textwidth}
  		\centering
  		\includegraphics[width=1\linewidth]{frec_srrc-crop.pdf}
  		\caption{Réponse fréquentielle}
  		\label{fig:srrc_frec1M}
	\end{subfigure}%
	\caption{Réponse impulsionnelles et fréquentielles du filtre SRRC - $D=1 Mbit/s ; \alpha =0.5$.}
	\label{fig:srrc1M}
\end{figure} 

%------------------15Mb
\begin{figure}
	\begin{subfigure}{.5\textwidth}
  		\centering
  		\includegraphics[width=1\linewidth]{impul_nrz_15-crop.pdf}
  		\caption{Réponse impulsionnelle}
  		\label{fig:nrz_impul15M}
	\end{subfigure}
	\begin{subfigure}{.5\textwidth}
  		\centering
  		\includegraphics[width=1\linewidth]{frec_nrz_15-crop.pdf}
  		\caption{Réponse fréquentielle}
  		\label{fig:nrz_frec15M}
	\end{subfigure}%
	\caption{Réponse impulsionnelles et fréquentielles du filtre NRZ - $D=15 Mbit/s$}
	\label{fig:nrz15M}
\end{figure} 

\begin{figure}
	\begin{subfigure}{.5\textwidth}
  		\centering
  		\includegraphics[width=1\linewidth]{impul_rz_15-crop.pdf}
  		\caption{Réponse impulsionnelle}
  		\label{fig:rz_impul15M}
	\end{subfigure}
	\begin{subfigure}{.5\textwidth}
  		\centering
  		\includegraphics[width=1\linewidth]{frec_rz_15-crop.pdf}
  		\caption{Réponse fréquentielle}
  		\label{fig:rz_frec15M}
	\end{subfigure}%
	\caption{Réponse impulsionnelles et fréquentielles du filtre RZ - $D=15 Mbit/s$}
	\label{fig:rz15M}
\end{figure}

\begin{figure}
	\begin{subfigure}{.5\textwidth}
  		\centering
  		\includegraphics[width=1\linewidth]{impul_srrc_15-crop.pdf}
  		\caption{Réponse impulsionnelle}
  		\label{fig:srrc_impul15M}
	\end{subfigure}
	\begin{subfigure}{.5\textwidth}
  		\centering
  		\includegraphics[width=1\linewidth]{frec_srrc_15-crop.pdf}
  		\caption{Réponse fréquentielle}
  		\label{fig:srrc_frec15M}
	\end{subfigure}%
	\caption{Réponse impulsionnelles et fréquentielles du filtre SRRC - $D=15 Mbit/s ; \alpha =0.5$.}
	\label{fig:srrc15M}
\end{figure}

\newpage

Maintenant on va parler sur les changements du coefficient d'excès de bande $\alpha$ pour les filtre SRRC. Dans les figures \ref{fig:srrc15M} et \ref{fig:srrc1M} il est toujours $0.5$ mais pour les figures \ref{fig:srrc1M_alp0} et \ref{fig:srrc1M_alp1} $\alpha$ est 0 et 1. Quand $\alpha=0$ la bande passante est la moitié que pour $\alpha = 1$ où $R=D$. ils sont les cas critique pour les filtres SRRC.       

%-------------------1M alpha 0
\begin{figure}[htb]
	\begin{subfigure}{.5\textwidth}
  		\centering
  		\includegraphics[width=1\linewidth]{impul_srrc_alpha_0-crop.pdf}
  		\caption{Réponse impulsionnelle}
  		\label{fig:srrc_impul1M_alp0}
	\end{subfigure}
	\begin{subfigure}{.5\textwidth}
  		\centering
  		\includegraphics[width=1\linewidth]{frec_srrc_alpha_0-crop.pdf}
  		\caption{Réponse fréquentielle}
  		\label{fig:srrc_frec15M_alp0}
	\end{subfigure}%
	\caption{Réponse impulsionnelles et fréquentielles du filtre SRRC - $D=1bit/s ; \alpha =0$.}
	\label{fig:srrc1M_alp0}
\end{figure}

%-------------------1M alpha 1
\begin{figure}[htb]
	\begin{subfigure}{.5\textwidth}
  		\centering
  		\includegraphics[width=1\linewidth]{impul_srrc_alpha_1-crop.pdf}
  		\caption{Réponse impulsionnelle}
  		\label{fig:srrc_impul1M_alp1}
	\end{subfigure}
	\begin{subfigure}{.5\textwidth}
  		\centering
  		\includegraphics[width=1\linewidth]{frec_srrc_alpha_1-crop.pdf}
  		\caption{Réponse fréquentielle}
  		\label{fig:srrc_frec1M_alp1}
	\end{subfigure}%
	\caption{Réponse impulsionnelles et fréquentielles du filtre SRRC - $D=1Mbit/s ; \alpha =1$.}
	\label{fig:srrc1M_alp1}
\end{figure}
  

\subsection{Mise en forme des symboles}

\subsubsection*{Question 3}

\begin{figure}[htb]
	\begin{center}
	\includegraphics[scale=1]{question3-crop.pdf}
	\caption{Mise en forme des symboles $a_k$}
	\label{fig:ques3}
	\end{center}
\end{figure} 

La figure \ref{fig:ques3} montre seulement la partie finale des signals $x_t$ et $s_t$ a fin de faire la comparaison de la durée. La signal $x_t$ est plus longue grâce à le filtrage de mise en forme et la durée du filtre. Dans notre cas le filtre a une durée de $8T_b$ et cela fait le décalage temporel du signal $x_t$ par rapport à $s_t$.  
%%----corecteur

\subsubsection*{Question 4}
La figure \ref{fig:ques4_st} montre l'spectre du signal $x_t$ et la figure \ref{fig:ques4_xt} pareil pour $s_t$. La signal modulé $x_t$ a toute la puissance dans la bande passante $R=D$, c'est la raison pour préférer ce type de modulation pour les transmissions de bande limitée.\\

Un autre estimateur de la densité spectral du puissance est la transforme de Fourier de l'autocorretation du signal, mais on ne l'implémente pas pour des raisons d'affichage.

\begin{figure}[htb]
	\begin{center}
	\includegraphics[scale=1]{welch_s_t-crop.pdf}
	\caption{Densité spectrale du puissance de $s_t$}
	\label{fig:ques4_st}
	\end{center}
\end{figure} 

\begin{figure}[htb]
	\begin{center}
	\includegraphics[scale=1]{welch_x_t-crop.pdf}
	\caption{Densité spectrale du puissance de $x_t$}
	\label{fig:ques4_xt}
	\end{center}
\end{figure} 


\subsubsection*{Question 5}
Quand les symboles sont non centrées, $x_t$ et $s_t$ dans la figure \ref{fig:ques5_q3}, la densité spectrale du puissance des signals $s_t$ et $x_t$ aura des pics comme on peut regarder dans la figure \ref{fig:ques5_Q4st} et \ref{fig:ques5_Q4xt}.

\begin{figure}[htb]
	\begin{center}
	\includegraphics[scale=1]{question3_noncentree-crop.pdf}
	\caption{Mise en forme des symboles $a_k$ non-centrés.}
	\label{fig:ques5_q3}
	\end{center}
\end{figure}

\begin{figure}
	\begin{center}
	\includegraphics[scale=1]{welch_s_t_noncentree-crop.pdf}
	\caption{Densité spectrale du puissance de $s_t$ non-centrée.}
	\label{fig:ques5_Q4st}
	\end{center}
\end{figure} 

\begin{figure}
	\begin{center}
	\includegraphics[scale=1]{welch_x_t_noncentree-crop.pdf}
	\caption{Densité spectrale du puissance de $x_t$ non-centrée.}
	\label{fig:ques5_Q4xt}
	\end{center}
\end{figure}

\section{Ajout du bruit blanc gaussien}
L'ajoute du bruit blanc gaussien dans un canal est inévitable. C'est pour cela que le paramètre RSB est très important et aide à mesurer la puissance du bruit dans le canal. Cette expression est:
\[ \frac{E_b}{N_0}=F\frac{\sigma ^{2}_x}{\sigma ^{2}_n}\]
Où, \[ \sigma ^{2}_n = F.N_0/T\]
\subsubsection*{Question 6}
Selon les notes du cours la formule de puissance et l'énergie de bit pour les symboles non-corrélés et centrés est:
\[ P_x=\sigma _x ^2.Ts\left \| h_e \right \|^2 \]
\[E_b=\sigma _x ^2.Ts\left \| h_e \right \|^2.T_b\]

Où, \[Ts\left \| h_e \right \|^2 = 1\]
Pour à la fin obtenir que,
\[E_b=\sigma _x ^2.T_b \]
Finalement, à partir du RSB on obtient la formule suivante:
\[\frac{E_b}{N_0}=\frac{\sigma _x ^2.T_b}{N_0} = \frac{F.\sigma _x ^2}{F.N_0/T} = 
F\frac{\sigma ^{2}_x}{\sigma ^{2}_n}
\]
Alors la valeur de la variance du signal \emph{s(t)} est:
\[\sigma _x ^2=E[a_k^2]-m_a^2= \frac{1}{2}(1^2)+\frac{1}{2}(1^2)-0=\frac{1}{4}\]
Pour cela, l'expression de la variance du bruit est:
\[\sigma _n ^2=F\frac{\sigma _x ^2}{(E_b/N_0)_{dB}}\]
\section{Conversion analogique - numérique}
Dans une chaîne de transmission, au récepteur arrive un signal bruité et des fois attenué. C'est pour cela que ce signal à besoin d'un certain traitement pour la convertir à numérique.

\subsection{Filtrage adapté}
\subsubsection*{Question 7}
Le filtre adapté est le filtre utilisé pour détecter un signal connu dans la présence de BBG. Son but est de maximiser la RSB en diminuant l'effet du bruit. En même temps le filtre adapté est aussi utile pour éliminer l'IES.
\subsubsection*{Question 8}
La forme du filtre adapté est la version inversé et decalé du filtre de mise en forme. C'est comme cela car on à besoin de maximiser la RSB. Cela a été fait via la commande \emph{fliplr}.
\begin{figure}
	\begin{center}
	\includegraphics[scale=0.5]{Q8-1-crop.pdf}
	\caption{Filtres de mise en forme et adapté}
	\label{fig:ques8-1}
	\end{center}
\end{figure} 
\begin{figure}
	\begin{center}
	\includegraphics[scale=0.5]{Q8-2-crop.pdf}
	\caption{Filtres de mise en forme et adapté}
	\label{fig:ques8-2}
	\end{center}
\end{figure} 
\begin{figure}
	\begin{center}
	\includegraphics[scale=0.5]{Q8-3-crop.pdf}
	\caption{Filtres de mise en forme et adapté}
	\label{fig:ques8-3}
	\end{center}
\end{figure} 


\subsubsection*{Question 9}
Comme on peut regarder dans les figures \ref{fig:ques8-1}, \ref{fig:ques8-2} et \ref{fig:ques8-3}, la sortie du filtre adapté est maximum dans le temps de bit et nul dans les multiples de ce temps, c'est pour cela qu'on vérifie qu'on peut transmettre sans IES.

\subsubsection*{Question 10}
Comme on peut regarder dans la figure \ref{fig:q10E}, c'est évident l'effet du bruit dans le diagramme de l'oeil, si on augmente la \emph{RSB} l'oeil sera plus ouvert. Pour la figure \ref{fig:q10a} on peut regarder que le diagramme change de façon que si on augmente le $\alpha$, l'oeil sera plus ouvert. Dans la figure \ref{fig:q10l} on observe que le paramètre \emph{L} n'affecte pas trop le diagramme de l'oeil. Finalement, dans la figure \ref{fig:q10t} l'effet du paramètre \emph{T} fait que les temps de symboles (où se trouve le maximum d'un symbole) soient décales.
\begin{figure}
	\begin{subfigure}{.5\textwidth}
  		\centering
  		\includegraphics[width=1\linewidth]{Q10-EbNo10-crop.pdf}
  		\caption{Diagramme de l'oeil pour RSB 10dB}
  		\label{fig:q10E10}
	\end{subfigure}
	\begin{subfigure}{.5\textwidth}
  		\centering
  		\includegraphics[width=1\linewidth]{Q10-EbNo20-crop.pdf}
  		\caption{Diagramme de l'oeil pour RSB 20dB}
  		\label{fig:q10E20}
	\end{subfigure}
		%
	\caption{Différentes diagrammes de l'oeil à différentes \emph{RSB}.}
	\label{fig:q10E}
\end{figure}

\begin{figure}
	\begin{subfigure}{.5\textwidth}
  		\centering
  		\includegraphics[width=1\linewidth]{Q10-alfa025-crop.pdf}
  		\caption{Diagramme de l'oeil pour $\alpha$ = 0.25}
  		\label{fig:q10a025}
	\end{subfigure}
	\begin{subfigure}{.5\textwidth}
  		\centering
  		\includegraphics[width=1\linewidth]{Q10-alfa075-crop.pdf}
  		\caption{Diagramme de l'oeil pour $\alpha$ = 0.75}
  		\label{fig:q10a075}
	\end{subfigure}
	\caption{Différentes diagrammes de l'oeil à différentes $\alpha$.}
	\label{fig:q10a}
\end{figure} 

\begin{figure}
	\begin{subfigure}{.5\textwidth}
  		\centering
  		\includegraphics[width=1\linewidth]{Q10-EbNo20-crop.pdf}
  		\caption{Diagramme de l'oeil pour L = 8}
  		\label{fig:q10l16}
	\end{subfigure}
	\begin{subfigure}{.5\textwidth}
  		\centering
  		\includegraphics[width=1\linewidth]{Q10-L64-crop.pdf}
  		\caption{Diagramme de l'oeil pour L = 64}
  		\label{fig:q10l64}
	\end{subfigure}
	\caption{Différentes diagrammes de l'oeil à différentes valeurs de L.}
	\label{fig:q10l}
\end{figure} 
\begin{figure}
	\begin{subfigure}{.5\textwidth}
  		\centering
  		\includegraphics[width=1\linewidth]{Q10-Tsur2-crop.pdf}
  		\caption{Diagramme de l'oeil pour T = $\frac{1}{2*D}$}
  		\label{fig:q10t2}
	\end{subfigure}
	\begin{subfigure}{.5\textwidth}
  		\centering
  		\includegraphics[width=1\linewidth]{Q10-EbNo20-crop.pdf}
  		\caption{Diagramme de l'oeil pour T = $\frac{1}{D}$}
  		\label{fig:q10t1}
	\end{subfigure}
	\caption{Différentes diagrammes de l'oeil à différentes $\alpha$.}
	\label{fig:q10t}
\end{figure} 



\subsection{Décimation}
\subsubsection*{Question 11}
Après le filtre adapté il faut prendre les valeurs $y_k$ qui conviennent. Comme c'était mentionné au-dessus le but du filtre adapté c'est de maximiser le \emph{RSB}, donc c'est pour cela qu'en ayant deux signaux de la même longueur, alors le résultat maximum de la convolution se trouvera à la valeur de la longueur. Dans ce cas sera tous les \emph{LT}. 
\subsubsection*{Question 12}
C'est très important d'avoir de la synchronisation dans un système de transmission, car si ce n'est pas comme cela on échantillonnera le signal reçu dans le temps invalide et cela puisse arriver parce qu'un canal à \emph{BBG} ajoute aussi un déphasage au signal. Finalement, si c'est le cas, le filtre adapté ne maximisera pas le \emph{RSB} dans le temps de symbole.\section{Prise de décision \emph{(demapping)}}
\subsubsection*{Question 13}
Dans la suite de la chaîne de communication, une fois que les symboles $y_k$ sont reçus il faut prendre un seuil de décision. À l'aide la figure \ref{fig:q13} on voit que le seuil optimal est en \emph{0}. 
\begin{figure}
	\begin{center}
	\includegraphics[scale=1]{Q13-crop.pdf}
	\caption{Constellation des $y_k$.}
	\label{fig:q13}
	\end{center}
\end{figure}  

\section{Calcul du taux d'erreur binaire}
Après le seuil optimal il faut comparer et trouver combien d'erreurs on a commit dans la transmission des symboles. Donc c'est avec le \emph{TEB} dans lequel on calcule le rapport entre le nombre d'éléments binaires faux et le nombre d'éléments binaires transmis.
\section{Mesures de performances}
Finalement, c'est important mesurer la chaîne de transmission. C'est fait avec la relation entre le \emph{RSB} et la probabilité d'erreur.
\subsubsection*{Question 14}
Dans la figure \ref{fig:q14} on trouve les deux courbes de \emph{BER}. Ici la courbe empirique arrive qu'à 7dB de \emph{EbNo} car pour pouvoir continuer c'est neccesaire d'envoyer plus qu'un million de bits et étant donné les limitations de mémoire c'était impossible. L'important est que c'est évident que la courbe empirique est pire que la théorique, quelque chose qui est attendu, car la courbe théorique est fait à partir d'estimations et méthodes statistiques qui ne prennent pas certains aspects de la chaîne de transmission. Enfin, la courbe empirique doit être toujours pire que la théorique sauf quand on utilise des codes détecteurs et correcteurs d'erreurs ainsi que de la codification de canal.

\begin{figure}
	\begin{center}
	\includegraphics[scale=1]{Q14-BER-crop.pdf}
	\caption{Constellation des $y_k$.}
	\label{fig:q14}
	\end{center}
\end{figure}  
%\begin{figure}[htb]
%\begin{center}
%\includegraphics[scale=0.5]{Q2total}
%\caption{TFD du signal complété par des zéros}
%\end{center}
%\end{figure}

\end{document}