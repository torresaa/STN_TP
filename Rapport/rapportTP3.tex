\documentclass[a4paper,11pt]{article}
	\usepackage[]{mcode}
    \usepackage[utf8]{inputenc}
    \usepackage[T1]{fontenc} 
    \usepackage[french]{babel}
    \usepackage{graphicx}
    \usepackage{anysize}
    \usepackage{amsfonts}
    \marginsize{2cm}{2cm}{2cm}{2cm}
    \title{Simulation d'une chaîne de transmission numérique sur canal gaussien à bande limitée\newline [Rapport]}
    \author{Jorge GULFO MONSALVE\\Aquiles TORRES ALVAREZ}

\begin{document}
\maketitle

\section{Géneration aléatoire des éléments binaires}



\section{Conversion des éléments binaires en symboles \emph{(mapping)}}

\section{Conversion numérique - analogique}

\subsection{Expansion}

\subsubsection*{Question 1}

\subsection{Étude des filtres}

\subsubsection*{Question 2}

\subsection{Mise en forme des symboles}

\subsubsection*{Question 3}
\subsubsection*{Question 4}
\subsubsection*{Question 5}

\section{Ajout du bruit blanc gaussien}
\subsubsection*{Question 6}

\section{Conversion analogique - numérique}
\subsection{Filtrage adapté}
\subsubsection*{Question 7}
\subsubsection*{Question 8}
\subsubsection*{Question 9}
\subsubsection*{Question 10}

\subsection{Décimation}
\subsubsection*{Question 11}
\subsubsection*{Question 12}

\section{Prise de décision \emph{(demapping)}}
\subsubsection*{Question 13}

\section{Calcul du taux d'erreur binaire}

\section{Mesures de performances}
\subsubsection*{Question 14}

%\begin{figure}[htb]
%\begin{center}
%\includegraphics[scale=0.5]{Q2total}
%\caption{TFD du signal complété par des zéros}
%\end{center}
%\end{figure}

\end{document}